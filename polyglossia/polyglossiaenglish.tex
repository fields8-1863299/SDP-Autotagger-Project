%!TEX TS-program = xelatex

\documentclass{article}

\usepackage{geometry}
\geometry{letterpaper}
\usepackage{setspace}

\usepackage{fontspec}
\usepackage{polyglossia}
\setmainlanguage{english}
\setotherlanguage{arabic}
\newfontfamily\arabicfont[Script = Arabic]{Al Nile} % Replace 'Simplified Arabic' with a font from your system

\begin{document}

Just plain Arabic:
\ \\

\begin{Arabic}

انطلقت الثورة الليبية منذ أكثر من سنة متوازية مع ثورات الربيع العربي الأخرى التي أطاحت ببعض حكام البلدان العربية، وما إن وافقت الجامعة العربية وأمريكا وأوروبا على ضرورة تدخل منظمة حلف شمال الأطلسي في المسألة حتى دخلت تركيا على الخط وقامت بإيصال مواد إغاثية وعسكرية دعما للثوار الليبيين. وهذا التدخل التركي يعكس الواقع الجديد في ملف العلاقات الخارجية بين تركيا والشرق الأوسط، ألا وهو تعاظم دور تركيا في العالم الناطق بلغة الضاد، حيث باتت تركيا تنتهج سياسة خارجية نشطة في القضايا الشائكة والعالقة في الشرق الأوسط. ومع هذا الاتجاه الجديد في سياسات تركيا، ينبغي تحديد معالم دور تركيا الجديد، ومحددات هذا الدور في ضوء التغيرات الأخيرة في المنطقة، وكذلك تحديد آراء العرب حيال الموقف التركي في الشرق الأوسط.

عليه فإن هذا ما أسعى وراءه في هذا البحث، حيث ساستعرض خلفية الدور التركي التاريخية والسياسية، وأحاول أن أحدد معالم هذا الدور في ضوء التطورات في الصراع العربي الإسرائيلي من ناحية وثورات الربيع العربي من ناحية أخرى، كما سأناقش آراء العرب الإيجابية والسلبية في الموضوع. علما أن غبار الربيع العربي لم يستقر بعد، وأن العلاقات التركية-الإسرائيلية تمر بمرحلة غامضة، من الصعب أن نتكهن بنتائجها، وعلى الرغم من ذلك أعتقد أن تركيا لديها أوراق في ثورات الربيع العربي وكذلك في مسألة الصراع العربي الإسرائيلي، التي ستفيد الطرفين على المدى البعيد وإن لم نر مؤشرات واضحة في المدى القصير.

\end{Arabic}
\ \\

To even the spacing between lines use the \verb|setspace| package:
\ \\

\begin{Arabic}

\setstretch{1.4}

انطلقت الثورة الليبية منذ أكثر من سنة متوازية مع ثورات الربيع العربي الأخرى التي أطاحت ببعض حكام البلدان العربية، وما إن وافقت الجامعة العربية وأمريكا وأوروبا على ضرورة تدخل منظمة حلف شمال الأطلسي في المسألة حتى دخلت تركيا على الخط وقامت بإيصال مواد إغاثية وعسكرية دعما للثوار الليبيين. وهذا التدخل التركي يعكس الواقع الجديد في ملف العلاقات الخارجية بين تركيا والشرق الأوسط، ألا وهو تعاظم دور تركيا في العالم الناطق بلغة الضاد، حيث باتت تركيا تنتهج سياسة خارجية نشطة في القضايا الشائكة والعالقة في الشرق الأوسط. ومع هذا الاتجاه الجديد في سياسات تركيا، ينبغي تحديد معالم دور تركيا الجديد، ومحددات هذا الدور في ضوء التغيرات الأخيرة في المنطقة، وكذلك تحديد آراء العرب حيال الموقف التركي في الشرق الأوسط.

عليه فإن هذا ما أسعى وراءه في هذا البحث، حيث ساستعرض خلفية الدور التركي التاريخية والسياسية، وأحاول أن أحدد معالم هذا الدور في ضوء التطورات في الصراع العربي الإسرائيلي من ناحية وثورات الربيع العربي من ناحية أخرى، كما سأناقش آراء العرب الإيجابية والسلبية في الموضوع. علما أن غبار الربيع العربي لم يستقر بعد، وأن العلاقات التركية-الإسرائيلية تمر بمرحلة غامضة، من الصعب أن نتكهن بنتائجها، وعلى الرغم من ذلك أعتقد أن تركيا لديها أوراق في ثورات الربيع العربي وكذلك في مسألة الصراع العربي الإسرائيلي، التي ستفيد الطرفين على المدى البعيد وإن لم نر مؤشرات واضحة في المدى القصير.

\end{Arabic}
\ \\

Arabic in the middle of English text using the \verb|\textarabic| command (\textarabic{بسم الله الرحمن الرحيم}).
\ \\

English in the middle of the \verb|Arabic| environment:
\ \\

\begin{Arabic}

اللغة العربية (Arabic) من أكثر لغات العالم انتشاراً.

\end{Arabic}
\ \\

To have it use your default English font, use the \verb|\textenglish| command:
\ \\

\begin{Arabic}

اللغة العربية (\textenglish{Arabic}) من أكثر لغات العالم انتشاراً.

\end{Arabic}


\end{document}








